\documentclass[11pt]{article}
\usepackage{amsmath,amssymb,amsthm,amsfonts,hyperref}

\title{Puck Predictions: Unraveling the NHL Game Forecasting Riddle}
\author{Jason Vasquez \and Dylan Skinner \and Jeff Hansen \and Benjamin McMullin}

\begin{document}

\maketitle

\begin{abstract}
    The goal of this project is simple: predict the outcomes of NHL games from any given state.
    As simple as the problem statement is, however, the solution is not so straightforward.
    To solve this problem, we will use a variety of machine learning techniques, including logistic regression,
    XGBoost, and ARIMA models. Additionally, we utilize a form of MCMC to simulate the outcomes of games from any given
    state. Our hypothesis is that we will be able to successfully predict the outcomes of NHL games with a high degree of accuracy
    using these tools.
\end{abstract}

\section{Problem Statement and Motivation}
% Your content for this section goes here

\section{Data}
% Your content for this section goes here
Our data came from the hockeyR Github repository\cite{hockeyR-data}. This repository contains an abundance of data about every NHL game
that has occured since the 2010-11 season. This data includes information about the events that transpire in a game (hits, shots, goals, etc.),
which teams are playing, who is on the ice, and the final score of the game.

\section{Methods}
% Your content for this section goes here

\section{Results}
% Your content for this section goes here

\section{Analysis}
% Your content for this section goes here

\section{Ethical Considerations}
% Your content for this section goes here

\section{Conclusions}
% Your content for this section goes here

% Bibliography
\bibliographystyle{plain}
\bibliography{references} % Replace 'references' with the name of your .bib file

\end{document}

