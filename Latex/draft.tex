\documentclass[11pt]{article}
\usepackage{amsmath,amssymb,amsthm,amsfonts,hyperref}

\title{Puck Predictions: Unraveling the NHL Game Forecasting Riddle}
\author{Jason Vasquez \and Dylan Skinner \and Jeff Hansen \and Benjamin McMullin}

\begin{document}

\maketitle

\begin{abstract}
    The goal of this project is simple: predict the outcomes of NHL games from any given state.
    As simple as the problem statement is, however, the solution is not so straightforward.
    To solve this problem, we will use a variety of machine learning techniques, including logistic regression,
    XGBoost, and ARIMA models. Additionally, we utilize a form of MCMC to simulate the outcomes of games from any given
    state. Our hypothesis is that we will be able to successfully predict the outcomes of NHL games with a high degree of accuracy
    using these tools.
\end{abstract}

\section{Problem Statement and Motivation}
% Your content for this section goes here

\section{Data}
% Your content for this section goes here
Our data came from the hockeyR Github repository\cite{hockeyR-data}. This repository contains an abundance of data about every NHL game
that has occured since the 2010-11 season. This data includes information about the events that transpire in a game (hits, shots, goals, etc.),
which teams are playing, who is on the ice, and the final score of the game. The data is stored in a series of {\tt .csv.gz} files, allowing for
easy access and manipulation.

Each game in a season is given a unique identifier ({\tt game\_id}), which is constant across all events in a game. Every event that occurs in a game
will be stored in the {\tt event\_type} column. There are 17 unique event types, including things such as game start, faceoff, shot, hit, and goal.
Most of these event types are not relevant to our analysis, so we remove them from the dataset. After removing the unnecessary events, we are left with
nine events: blocked shot, faceoff, giveaway, goal, hit, missed shot, penalty, shot, and takeaway. These events are attributed to the
team and player that performs the event. We only take into consideration the team that performs the event and discard the player information.

The data also contains information about when the event occured. This appears in a variaty of formats, but we only
use the {\tt game\_time\_remaining} column. {\tt game\_time\_remaining} starts
at 3600 (60 minutes) and counts down to 0. If the game goes into extra time, i.e., it is tied after 60 minutes, {\tt game\_time\_remaining} will
be a negative value.

We found that our data did not contain any missing values that was not easily explainable. For example, if a game is starting, there will be no
events for penalties, which will result in a {\tt NaN} value in the penalties column. Additionally, any data that was confusing or not easily explainable
(for example the home team having 7 players on the ice and the away team having 5), was manually verified by watching a clip of the game where
the event occured to make sure the event was recorded correctly. We did not find any incorrectly recorded events, so we 
did not remove any strange events from out dataset.

\section{Methods}
\subsection{t-SNE, UMAP, and PCA}

\subsection{Regression and XGBoost}

\subsection{MCMC Game Simulation}

\subsection{Exponential Smoothing}
% Your content for this section goes here

\section{Results}
% Your content for this section goes here

\section{Analysis}
% Your content for this section goes here

\section{Ethical Considerations}
% Your content for this section goes here

\section{Conclusions}
% Your content for this section goes here

% Bibliography
\bibliographystyle{plain}
\bibliography{references} % Replace 'references' with the name of your .bib file

\end{document}

